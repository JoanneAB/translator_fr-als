% ---------------------------------------------------------------------------------------------------------------
\section{Conclusion}
This project presents the work undertaken to generate a model for a translation task from French to Alsatian 
languages. Results show that the models are always over-fitting the data, showing that the compiled dataset is 
not sufficient to fine-tuned a model to get an accurate and robust translation for words and sentences that are 
not well represented in the dataset. Indeed, only 54\% of the proposed translations to some test participants 
were considered as a good translation while more than 25\% where not understandable. Improving the dataset by 
adding more data, inserting labels/contexts to each entry or manualy checking the entries could improve the 
fine-tuning that would lead to better accuracy of the translations.

Because Alsatian is a spoken and non-standard language with significant regional variation across Alsace, the 
development of a robust and accurate models was not expected. The objective of this project was to explore and 
apply the concepts learned during the course on a challenging application. I wanted to better understand the 
fine-tuning process and the importance of having a good understanding of both the data and the model parameters 
in order to build confidence in the resulting model.

